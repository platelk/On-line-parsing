Parsers are intensely use in many software and application for different purpose, one of this purpose is to provide useful and efficient software development tools and it's important for developers productivity to have tools that allow them to make a quick modification even inside important file as configuration file for example.\\
This research paper focus on the efficient aspect of parsers that are capable to react at each input (also called 'on-line' parser) without losing to much help features (auto-completion, error detection, ...) by analysing 2 big family of parsers, Bottom-Up and Top-down, and will look deeper in a possible combination of this 2 approach of parsing.\\
The global approach is that in the edition of important document section, the bottom-up approach can provide good performance where its the most needed, but the top-down approach can be better for little modifications and "hand-writing" code.\\
Implementing this kind of algorithm can be difficult, but it may open the research on tools or parsers to a new approach of algorithm combination and "context-aware" parsing.