In the section, i will describe more deeply the algorithm i want to implement and the choice behind
\subsubsection{Taking care of the context}
One most important variable to take in count is the context where the algorithm will be run. On the context will depend the "mode" of the algorithm, actually 3 context will be treated :
\begin{itemize}
\item Opening the file: \\
This context is during the first read of the all file
\item small edition in the file ( \( buffer < 500 character\) )\\
This context is mostly during a small copy\/paste or during simple edition (keystroke)
\item Important edition in the file ( \( buffer >= 500 character\) )\\
This context is mostly during big copy\/paste
\end{itemize}

To know this context every time, i will have :

\begin{itemize}
\item Record where am i in the file ( character number n, line number l )
\item Update the position of each node inside the syntax tree every time a modification are made
\item Record the number of pending character inside the reading buffer
\end{itemize}
This will allow me to know which part of the file have been modify, and where the modification has been made.\\

\subsubsection{Parse the smallest amount of data}
An other part of the algorithm will be to limit the input treated by the algorithm. An easy solution when the user insert a new character is to apply the parser on the all line, but in some case it's not necessary and maybe only a part of the line can be re-parser.\\
example :
\begin{lstlisting}[language=C++, caption=parsing optimisation in C]
// inserting 2 here
//           |
//           v
int a = 9 * 82 / 53;
\end{lstlisting}
TODO : create schema of a syntax tree of this expression \\
in this expression, the syntactic tree will show that if a modification happen between character '*' and '\/', the node impacted is '8', so before run the parser on the all line, the algorithm will run the parser on the smallest sub-syntactic tree possible to know if it's possible to resolve the input.

\subsubsection{Running the right parsing algorithm}
Knowing in which context i am, the aims will be to run the top-down or the bottom-up algorithm depend of the context.
\begin{itemize}
\item If it's the first read of the file or during a important edition : run the bottom-up
\item If it's a small edition : run the top-down
\end{itemize}

The first read will create the general tree, then tree created during the top-down will be insert inside the main tree and during important modification, the all file will be re-parse.\\
The choice to re-parse the all file during important modification is made because implement a more clever algorithm will be to much time consuming but other technique exist, as try to create a sub-tree with a bottom-up approach and then try to insert the sub-tree inside the main tree

% TODO : create syntax tree of this expression
